% \documentclass[a4paper,fleqn,leqno]{article}
\documentclass[a4paper]{article}
\usepackage{enumerate}
\usepackage{blindtext}

\usepackage{amsmath}
\usepackage[mathcal,mathscr]{eucal}
\usepackage{eufrak}
\usepackage{amssymb}

\begin{document}
    \section{Mathematics}
    \subsection{Inline formulas}
    Inline formula:
    $a^2+b^2=\sqrt{c}$. The formula is part of text. Another possibility: 
    \(\int_a^b\sin(x)dx\). Yet another possibility:
    \begin{math}
        \aleph
    \end{math}
    All three are used to be shown in text.

    \subsection{Display formulas}
    Displayed formula:
    $$a^2+b^2=\sqrt[b]{c}$$ The formula is typeset to the next line. Another possibility: 
    \[\int_a^b\sin(x)dx\] Yet another possibility:
    \begin{displaymath}
        \aleph
    \end{displaymath}
    All three are used to display not in text, but seperatly.
    \ref{squareRoot} % \eqref{squareRoot}

    \subsection{Display and numbered formulas}
    \begin{equation}
        1+2+3+\sqrt{2} \label{squareRoot}
    \end{equation}

    \begin{eqnarray}
        B&===&\frac{\partial f(x,y)\cdot g(x,y)}{\partial x}\\
        B(x,y)&=&\frac{\partial \vec{f}(x,y)
        \times\vec{g}(x,y)^{(2+3)}}{\partial x_{12+1}}
    \end{eqnarray}

    \subsection{Function in math}
    $\sin(\alpha)$
    $\cos(\beta)$
    $\log(x)$
    $\exp(\infty)$

    \begin{eqnarray}
        u_d\to o(1/R)&\text{as}&R\to\infty\nonumber\\
        \partial_n U_D\to o(1/R)&\text{as}&R\to\infty
    \end{eqnarray}

    \subsection{Mathematical fonts}
    $\dot{x}$, $\ddot{x}$, $\hat{A}$, $\vec{v}$

    $\mathcal{ABCDEF}$

    $\mathscr{ABCDEF}$

    $\mathfrak{ABCDEF}$

    $\mathbb{NZQRC}$

    \subsection{The amsmath package}
    \begin{multline}
        a=3b+6d=\\
        \shoveleft{=3\cdot 2c+6d=}\\
        =3\cdot 2c+6d=\\
        =6c+6d\\
        1m2e3o4w\\
        =6c+12e
    \end{multline}

    \[
    \begin{split}
        A&=25\cdot10^{23}\\
        \Theta&=300
    \end{split}
    \]

    \begin{gather}
        \sin^2x+\cos^2x=1\\
        \frac{\sin x}{\cos x}=\tan x
    \end{gather}

    \begin{equation}
        \begin{gathered}
            \sin^2x+\cos^2x=1\\
            \frac{\sin x}{\cos x}=\tan x
        \end{gathered}
    \end{equation}

    \[
    \left.
    \begin{gathered}
        3x+5y=15\\
        2x-4y=20
    \end{gathered}\right\} \Longrightarrow
    \begin{gathered}
        x=\frac{80}{11}\\
        y=-\frac{15}{11}
    \end{gathered}
    \]

    \begin{align}
        1&=1&2&=2&2&=1+1\\
        3+2&=5&3&=2+1&3&=1+1+1\\
    \end{align}

    \[
    \left.
    \begin{aligned}
        3x+5y=15\\
        2x-4y=20
    \end{aligned}\right\} \Longrightarrow
    \begin{aligned}
        x=\frac{80}{11}\\
        y=\frac{15}{11}
    \end{aligned}
    \]

    \begin{gather}
        x=ac+bc\\
        y>dc\notag\\
        e^{i\pi}+1=0\tag{Euler}\\
        e^{i\pi}+1=0\tag*{E-1}
    \end{gather}

    \begin{gather}
        x=ac+bc\label{FuckMeInTheAssTonight}\\
        X*ac=+bc+1\tag{\ref{FuckMeInTheAssTonight}$^*$}
    \end{gather}

    \begin{subequations}
        \begin{eqnarray}
            a=1\label{makeItUnique}\\
            b_{\mbox{text}}=1+1\\
            c_{\text{text}}=1+1+1\\
            d=1+2
        \end{eqnarray}
    \end{subequations}

    \ref{makeItUnique}
    \eqref{makeItUnique}

    \[
    \sum_{a=R_a}^{\sum B}\frac{\sum a}{\int}
    \]

    \[
    \textstyle\sum_{a=R_a}^{\displaystyle\sum B}\frac{\scriptscriptstyle\sum a}{\scriptstyle\int}
    \]

    $\displaystyle\int$

    \subsection{Sub- and superscript}
    $a_{bc}$, $a\sb{bc}$
    $a^{bc}$, $a\sp{bc}$
    $\sideset{_a^b}{_c^d}\prod$

    $a_{\text{Some Text}}$\\
    $a_{\mbox{Some Text}}$\\
    $a_{\mathrm{Some Text}}$\\
    $a_{\textrm{Some Text}}$

    $a b c d e$

\end{document}