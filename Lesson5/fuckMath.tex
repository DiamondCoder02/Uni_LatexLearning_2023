\pagebreak
\section{Mathematics}
\subsection{Inline formulas}
Inline formula:
$a^2+b^2=\sqrt{c}$. The formula is part of text. Another possibility: 
\(\int_a^b\sin(x)dx\). Yet another possibility:
\begin{math}
    \aleph
\end{math}
All three are used to be shown in text.

\subsection{Display formulas}
Displayed formula:
$$a^2+b^2=\sqrt[b]{c}$$ The formula is typeset to the next line. Another possibility: 
\[\int_a^b\sin(x)dx\] Yet another possibility:
\begin{displaymath}
    \aleph
\end{displaymath}
All three are used to display not in text, but seperatly.
\ref{squareRoot} % \eqref{squareRoot}

\subsection{Display and numbered formulas}
\begin{equation}
    \sqrt{2} \label{squareRoot}
\end{equation}

\begin{eqnarray}
    B&===&\frac{\partial f(x,y)\cdot g(x,y)}{\partial x}\\
    B(x,y)&=&\frac{\partial \vec{f}(x,y)
    \times\vec{g}(x,y)^{(2+3)}}{\partial x_{12+1}}
\end{eqnarray}

$\sum_{i=1}^{\infty} i$\\
$\displaystyle\sum_{i=1}^{\infty} i$
$$\sum_{i=1}^{\infty} i$$

$\sum_{1}^{\infty} i$
$$\sum_{1}^{\infty} i$$
$$\intop_{1}^{\infty} i$$